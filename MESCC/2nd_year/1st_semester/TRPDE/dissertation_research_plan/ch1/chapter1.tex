\chapter{Introduction}
\label{chap:Chapter1}
%-------------------------------------------------------------------------------%
\glspl{uav} have witnessed a surge in popularity and research attention.
They have become indispensable in various applications, ranging from surveillance to reconnaissance, due to their versatility and efficiency in various applications \cite{UAV1}.

This growing interest is evident in the numerous review papers exploring different aspects of \gls{uav} development, ranging from open-source hardware and software utilization \cite{UAV2}, \cite{UAV3},\cite{UAV4}, frame design and optimization \cite{UAV5}, \cite{UAV6}, control systems, including both conventional and modern communication modalities such as 5G networks \cite{UAV7}, \cite{UAV8}, \cite{UAV10}, \cite{UAV11}, to efficient power management strategies, and alternative energy sources to extend \gls{uav} battery life \cite{UAV15}.\\

Nowadays, most \gls{VTOL} \glspl{uav}, rely on proprotors with fixed pitch systems because of their simplicity and lack of better \gls{COTS} reliable and efficient solutions but they impose limitations on the achievable flight performance \cite{FPP1}.
Thrust generation is confined to a single direction, hindering the \gls{uav}'s ability to produce upward thrust relative to the vehicle body.
Additionally, the control bandwidth is restricted by the inertia of the motors and proprotors, constraining the \gls{uav}'s agility and maneuverability \cite{FPP1}.\\
As described in recent studies \cite{FPP1}, these limitations become more pronounced as \gls{uav} size increases, impacting stability and control.
Larger \glspl{uav} face challenges as the need for larger motors with higher inertia compromises rapid control through \gls{RPM} adjustments alone.\\

The development of Variable Pitch Proprotor (VPP) systems plays a crucial role in overcoming the limitations of traditional \gls{uav} designs, such as those with fixed pitch proprotors \cite{VPP1}, \cite{FPP1}.
Several detailed descriptions of quadrotors, for instance, modeling and dynamics have been published \cite{FPP2}, \cite{FPP3}, \cite{FPP4}, \cite{FPP5}, emphasizing the need for dynamic control mechanisms.\\
The use of VPP systems, especially in \gls{VTOL} \glspl{uav}, addresses challenges related to control instabilities and energy efficiency \cite{VPP1}.\\
The incorporation of variable-pitch proprotors provides the necessary flexibility to enhance stability and enable larger \glspl{uav} to perform sophisticated maneuvers, overcoming the constraints inherent in fixed-pitch designs \cite{VPP1}, \cite{FPP1}.

TODO: MORE INFO



\section{Problem Analyses}
The utilization of fixed-pitch propellers in \glspl{uav} presents a set of limitations that significantly impact aircraft performance and efficiency.
These issues are evident in various phases of flight, including hover and forward flight, and have repercussions for the \gls{uav}.

TODO: MORE INFO\\

\subsection{Maneuverability and Response}
Fixed-pitch propellers inherently constrain \glspl{uav} from adjusting the pitch angle during flight \cite{FPP1}.
This limitation results in compromised maneuverability and response, restricting the range of aerobatic maneuvers that a \gls{uav} can execute \cite{FPP1}.
Additionally, the inability to change the pitch angle impedes the optimization of lift, landing, and thrust during flight, leading to sub-optimal performance in various operational scenarios \cite{FPP1}.

TODO: MORE INFO\\

\subsection{Power Consumption}
With fixed-pitch propellers, the power consumption will be higher.
Without the ability to adjust the propeller angle, \glspl{uav} may be forced to operate at higher \gls{RPM}s to compensate for this lack of adjustment \cite{FPP1}.
This higher power consumption not only affects the efficiency of the \gls{uav} but also has implications for its endurance, limiting the time the vehicle can remain airborne.\\

TODO: MORE INFO\\

\section{Motivation}
TODO: MORE INFO\\

\section{Objectives}
This thesis will focus on developing a stand-alone variable-pitch proprotor system that can, in real-time, change the propeller pitch according to each flight phase.

The first goal will be the understanding of the relevant fundamentals regarding \gls{VTOL} flight performance and the impact of variable proprotors.

Next, understanding the fundamentals of control subsystems, power management and short-range wireless communication protocols with particular emphasis on their reliability and real-time sensing and actuation.
There will also be made a survey about current solutions, communication technologies, electronic control and power management strategies.

After the research, it will be designed the subsystems architecture and then implemented the envisaged subsystem over a real mechanical prototype.

Finally, the performance of the system will be evaluated together with its limitations under different settings and environments.
