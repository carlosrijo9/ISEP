
% we include the glossary here (frontmatter is included with \input, so this command is as if it was in main.tex)
%All acronyms must be written in this file.
\newacronym{CPU}{CPU}{Central Processing Units}
\newacronym{COTS}{COTS}{Commercial Off-The Shelf}
\newacronym{FPGA}{FPGA}{Field-Programmable Gate Arrays}
\newacronym{GNSS}{GNSS}{Global Navigation Satellite System}
\newacronym{PCB}{PCB}{Pinted Circuit Board}
\newacronym{PWM}{PWM}{Pulse Width Modulation}
\newacronym{OBC}{OBC}{On Board Computer}
\newacronym{uav}{UAV}{Unmanned Aerial Vehicle}
\newacronym{RPM}{RPM}{Revolutions Per Minute}
\newacronym{RTC}{RTC}{Real-Time Clock}
\newacronym{RTOS}{RTOS}{Real-Time Operating System}
\newacronym{SoC}{SoC}{State of Charge}
\newacronym{UVP}{UVP}{Under Voltage Protection}
\newacronym{VTOL}{VTOL}{Vertical Take-Off and Landing}
\newacronym{EEPROM}{EEPROM}{Electrically Erasable Programmable Read-Only Memory}
\newacronym{SRAM}{SRAM}{Static Random Access Memory}
\newacronym{DRAM}{DRAM}{Dynamic Random Access Memory}
\newacronym{SD}{SD}{Secure Digital}
\newacronym{FRAM}{FRAM}{Ferroelectric Random Access Memory}
\newacronym{RAM}{RAM}{Random Access Memory}
\newacronym{SLC}{SLC}{Single-Level Cell}
\newacronym{MLC}{MLC}{Multi-Level Cell}
\newacronym{TLC}{TLC}{Triple-Level Cell}
\newacronym{eMMC}{eMMC}{embedded MultiMedia Card}
\newacronym{SSD}{SSD}{Solid State Drives}
\newacronym{HDD}{HDD}{Hard Disk Drives}
\newacronym{Li-ion}{Li-ion}{Lithium-ion}
\newacronym{Li-Po}{Li-Po}{Lithium Polymer}
\newacronym{NiMH}{NiMH}{Nickel–Metal Hydride}
\newacronym{NiCd}{NiCd}{Nickel-Cadmium}
\newacronym{ROS}{ROS}{Robot Operating System}
\newacronym{DDS}{DDS}{Data Distribution Service}
\newacronym{I2C}{I2C}{Inter-Integrated Circuit}
\newacronym{SPI}{SPI}{Serial Peripheral Interface}
\newacronym{UART}{UART}{Universal Asynchronous Receiver-Transmitter}
\newacronym{BLE}{BLE}{Bluetooth Low Energy}
\newacronym{TLS}{TLS}{Transport Layer Security}
\newacronym{Wifi}{Wifi}{Wireless Fidelity}
\newacronym{FIFO}{FIFO}{First In First Out}
\newacronym{USB}{USB}{Universal Serial Bus}
\newacronym{CAN}{CAN}{Controller Area Network}
\newacronym{RFID}{RFID}{Radio Frequency Identification}
\newacronym{NFC}{NFC}{Near Field Communication}
\newacronym{HDL}{HDL}{Hardware Description Languages}
\newacronym{ADC}{ADC}{Analog Digital Converter}
\newacronym{IoT}{IoT}{Internet of Things}
\newacronym{IEEE}{IEEE}{Institute of Electrical and Electronics Engineers}
\newacronym{LoRaWAN}{LoRaWAN}{Long Range Wide Area Network}


\frontmatter % Use roman page numbering style (i, ii, iii, iv...) for the pre-content pages

\pagestyle{plain} % Default to the plain heading style until the thesis style is called for the body content

%----------------------------------------------------------------------------------------
%	TITLE PAGE
%----------------------------------------------------------------------------------------

\maketitlepage


%----------------------------------------------------------------------------------------
%	ABSTRACT PAGE
%----------------------------------------------------------------------------------------

\begin{abstract}

% here you put the abstract in the main language of the work.

This document contains the main formatting rules to be applied in the writing of the report of the Thesis / Dissertation / Internship, of the Master in Critical Computing Systems Engineering, of the Department of Computer Engineering, of ISEP. The rules presented here form a set of best practices recommended for writing a dissertation. However, it is recommended to discuss these and other aspects with the respective supervisor. An annex provides some guidelines on adapting the template for the thesis research plan.

The rules to be followed are presented regarding the format of the paper to be used, how the document is organized, general rules for formatting the text, formatting tables and figures, inserting bibliographic references in the text and presenting bibliographic references.

The document must contain an abstract in Portuguese and English. The abstract in the  language of the document should come first and not exceed 200 words or 1 A4 page. The abstract in the other language should be an extended one, not exceeding 1000 words or 2 A4 pages.

This document was adapted from the master's dissertation model in Informatics Engineering at ISEP, originally prepared by Professor Fátima Rodrigues (DEI/ISEP).

After the abstract, it is mandatory to place the main keywords of the theme of the work, with a maximum of 6 keywords being allowed. Keywords are defined in the \emph{THESIS INFORMATION} block of the \file{main.tex} file.

\end{abstract}

\begin{abstractotherlanguage}
% here you put the abstract in the "other language": English, if the work is written in Portuguese; Portuguese, if the work is written in English.

Este documento contém as principais regras de formatação a aplicar na redação do relatório de Tese/Dissertação/Estágio, do Mestrado em Engenharia de Sistemas Computacionais Críticos, do Departamento de Engenharia Informática, do ISEP. As regras aqui apresentadas formam um conjunto de boas práticas recomendadas para escrever uma dissertação. No entanto, é recomendável discutir esses e outros aspetos com o respetivo supervisor. Em anexo é fornecido um guia para adaptar o conteúdo do documento para o plano de investigação de tese.

São apresentadas as regras a serem seguidas quanto ao formato a ser utilizado, a organização do documento, as regras gerais de formatação do texto, formatação de tabelas e figuras, inserção de referências bibliográficas no texto e apresentação de referências bibliográficas.

O documento deve conter o resumo em português e em inglês. O resumo no idioma do documento deve vir primeiro e não deve exceder 200 palavras ou 1 página A4. O resumo no outro idioma deve ser estendido, não excedendo 1000 palavras ou 2 páginas A4.

Este documento foi adaptado do modelo de dissertação de mestrado em Engenharia Informática do ISEP, originalmente elaborado pela Professora Fátima Rodrigues (DEI/ISEP).

Para alterar a língua basta ir às configurações do documento no ficheiro \file{main.tex} e alterar para a língua desejada ('english' ou 'portuguese')\footnote{Alterar a língua requer apagar alguns ficheiros temporários; O target \keyword{clean} do \keyword{Makefile} incluído pode ser utilizado para este propósito.}. Isto fará com que os cabeçalhos incluídos no template sejam traduzidos para a respetiva língua.

\end{abstractotherlanguage}


%----------------------------------------------------------------------------------------
%	LIST OF CONTENTS/FIGURES/TABLES PAGES
%----------------------------------------------------------------------------------------

\tableofcontents % Prints the main table of contents

\listoffigures % Prints the list of figures

\listoftables % Prints the list of tables

\listofalgorithms % Prints the list of algorithms
\addchaptertocentry{\listalgorithmname}


\renewcommand{\lstlistlistingname}{List of Source Code}

\lstlistoflistings % Prints the list of listings (programming language source code)
\addchaptertocentry{\lstlistlistingname}


%----------------------------------------------------------------------------------------
%	ABBREVIATIONS
%----------------------------------------------------------------------------------------
\begin{abbreviations}{ll} % Include a list of abbreviations (a table of two columns)
\textbf{MDU} & \textbf{M}ain \textbf{D}evice \textbf{U}nit\\
\textbf{SDU} & \textbf{S}econdary \textbf{D}evice \textbf{U}nit\\
\textbf{VPP} & \textbf{V}ariable \textbf{P}itch \textbf{P}roprotor\\
\textbf{FPP} & \textbf{F}ixed \textbf{P}itch \textbf{P}roprotor\\


\end{abbreviations}

%----------------------------------------------------------------------------------------
%	SYMBOLS
%----------------------------------------------------------------------------------------

% \begin{symbols}{lll} % Include a list of Symbols (a three column table)

% % [Note: Although acronyms and symbols are defined in this section, they should also be defined at least the first time used in the dissertation body.]

% $a$ & distance & \si{\meter} \\
% $P$ & power & \si{\watt} (\si{\joule\per\second}) \\
% %Symbol & Name & Unit \\

% \addlinespace % Gap to separate the Roman symbols from the Greek

% $\omega$ & angular frequency & \si{\radian} \\

% \end{symbols}



%----------------------------------------------------------------------------------------
%	ACRONYMS
%----------------------------------------------------------------------------------------

\newcommand{\listacronymname}{List of Acronyms}

%Use GLS
\glsresetall
\printglossary[title=\listacronymname,type=\acronymtype,style=long]

%----------------------------------------------------------------------------------------
%	DONE
%----------------------------------------------------------------------------------------

\mainmatter % Begin numeric (1,2,3...) page numbering
\pagestyle{thesis} % Return the page headers back to the "thesis" style
