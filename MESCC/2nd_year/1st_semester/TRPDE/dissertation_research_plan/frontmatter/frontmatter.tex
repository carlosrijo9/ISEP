
% we include the glossary here (frontmatter is included with \input, so this command is as if it was in main.tex)
%All acronyms must be written in this file.
\newacronym{CPU}{CPU}{Central Processing Units}
\newacronym{COTS}{COTS}{Commercial Off-The Shelf}
\newacronym{FPGA}{FPGA}{Field-Programmable Gate Arrays}
\newacronym{GNSS}{GNSS}{Global Navigation Satellite System}
\newacronym{PCB}{PCB}{Pinted Circuit Board}
\newacronym{PWM}{PWM}{Pulse Width Modulation}
\newacronym{OBC}{OBC}{On Board Computer}
\newacronym{uav}{UAV}{Unmanned Aerial Vehicle}
\newacronym{RPM}{RPM}{Revolutions Per Minute}
\newacronym{RTC}{RTC}{Real-Time Clock}
\newacronym{RTOS}{RTOS}{Real-Time Operating System}
\newacronym{SoC}{SoC}{State of Charge}
\newacronym{UVP}{UVP}{Under Voltage Protection}
\newacronym{VTOL}{VTOL}{Vertical Take-Off and Landing}
\newacronym{EEPROM}{EEPROM}{Electrically Erasable Programmable Read-Only Memory}
\newacronym{SRAM}{SRAM}{Static Random Access Memory}
\newacronym{DRAM}{DRAM}{Dynamic Random Access Memory}
\newacronym{SD}{SD}{Secure Digital}
\newacronym{FRAM}{FRAM}{Ferroelectric Random Access Memory}
\newacronym{RAM}{RAM}{Random Access Memory}
\newacronym{SLC}{SLC}{Single-Level Cell}
\newacronym{MLC}{MLC}{Multi-Level Cell}
\newacronym{TLC}{TLC}{Triple-Level Cell}
\newacronym{eMMC}{eMMC}{embedded MultiMedia Card}
\newacronym{SSD}{SSD}{Solid State Drives}
\newacronym{HDD}{HDD}{Hard Disk Drives}
\newacronym{Li-ion}{Li-ion}{Lithium-ion}
\newacronym{Li-Po}{Li-Po}{Lithium Polymer}
\newacronym{NiMH}{NiMH}{Nickel–Metal Hydride}
\newacronym{NiCd}{NiCd}{Nickel-Cadmium}
\newacronym{ROS}{ROS}{Robot Operating System}
\newacronym{DDS}{DDS}{Data Distribution Service}
\newacronym{I2C}{I2C}{Inter-Integrated Circuit}
\newacronym{SPI}{SPI}{Serial Peripheral Interface}
\newacronym{UART}{UART}{Universal Asynchronous Receiver-Transmitter}
\newacronym{BLE}{BLE}{Bluetooth Low Energy}
\newacronym{TLS}{TLS}{Transport Layer Security}
\newacronym{Wifi}{Wifi}{Wireless Fidelity}
\newacronym{FIFO}{FIFO}{First In First Out}
\newacronym{USB}{USB}{Universal Serial Bus}
\newacronym{CAN}{CAN}{Controller Area Network}
\newacronym{RFID}{RFID}{Radio Frequency Identification}
\newacronym{NFC}{NFC}{Near Field Communication}
\newacronym{HDL}{HDL}{Hardware Description Languages}
\newacronym{ADC}{ADC}{Analog Digital Converter}
\newacronym{IoT}{IoT}{Internet of Things}
\newacronym{IEEE}{IEEE}{Institute of Electrical and Electronics Engineers}
\newacronym{LoRaWAN}{LoRaWAN}{Long Range Wide Area Network}


\frontmatter % Use roman page numbering style (i, ii, iii, iv...) for the pre-content pages

\pagestyle{plain} % Default to the plain heading style until the thesis style is called for the body content

%----------------------------------------------------------------------------------------
%	TITLE PAGE
%----------------------------------------------------------------------------------------

\maketitlepage


%----------------------------------------------------------------------------------------
%	ABSTRACT PAGE
%----------------------------------------------------------------------------------------

\begin{abstract}

% here you put the abstract in the main language of the work.

TODO - abstract up to 200 words

\end{abstract}

\begin{abstractotherlanguage}
% here you put the abstract in the "other language": English, if the work is written in Portuguese; Portuguese, if the work is written in English.

TODO - abstract up to 1000 words ???
%\footnote{Alterar a língua requer apagar alguns ficheiros temporários}.
\end{abstractotherlanguage}


%----------------------------------------------------------------------------------------
%	LIST OF CONTENTS/FIGURES/TABLES PAGES
%----------------------------------------------------------------------------------------

\tableofcontents % Prints the main table of contents

\listoffigures % Prints the list of figures

\listoftables % Prints the list of tables

\listofalgorithms % Prints the list of algorithms
\addchaptertocentry{\listalgorithmname}


\renewcommand{\lstlistlistingname}{List of Source Code}

\lstlistoflistings % Prints the list of listings (programming language source code)
\addchaptertocentry{\lstlistlistingname}


%----------------------------------------------------------------------------------------
%	ABBREVIATIONS
%----------------------------------------------------------------------------------------
\begin{abbreviations}{ll} % Include a list of abbreviations (a table of two columns)
\textbf{MDU} & \textbf{M}ain \textbf{D}evice \textbf{U}nit\\
\textbf{SDU} & \textbf{S}econdary \textbf{D}evice \textbf{U}nit\\
\textbf{VPP} & \textbf{V}ariable \textbf{P}itch \textbf{P}roprotor\\
\textbf{FPP} & \textbf{F}ixed \textbf{P}itch \textbf{P}roprotor\\


\end{abbreviations}

%----------------------------------------------------------------------------------------
%	SYMBOLS
%----------------------------------------------------------------------------------------

% \begin{symbols}{lll} % Include a list of Symbols (a three column table)

% % [Note: Although acronyms and symbols are defined in this section, they should also be defined at least the first time used in the dissertation body.]

% $a$ & distance & \si{\meter} \\
% $P$ & power & \si{\watt} (\si{\joule\per\second}) \\
% %Symbol & Name & Unit \\

% \addlinespace % Gap to separate the Roman symbols from the Greek

% $\omega$ & angular frequency & \si{\radian} \\

% \end{symbols}



%----------------------------------------------------------------------------------------
%	ACRONYMS
%----------------------------------------------------------------------------------------

\newcommand{\listacronymname}{List of Acronyms}

%Use GLS
\glsresetall
\printglossary[title=\listacronymname,type=\acronymtype,style=long]

%----------------------------------------------------------------------------------------
%	DONE
%----------------------------------------------------------------------------------------

\mainmatter % Begin numeric (1,2,3...) page numbering
\pagestyle{thesis} % Return the page headers back to the "thesis" style
