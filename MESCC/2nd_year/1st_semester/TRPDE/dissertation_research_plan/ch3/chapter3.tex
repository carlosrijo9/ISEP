\chapter{Technologies}
\label{chap:Chapter3}
%-------------------------------------------------------------------------------%

\section{Firmware}
In embedded systems, the choice of programming languages and the use of a \gls{RTOS} in firmware development are critical decisions that can impact the performance, efficiency, and complexity of the embedded system.\\

As for Programming Languages, there is \textit{C} widely used in embedded systems due to its low-level features and close to the hardware, efficient use of system resources, and strong support from the embedded development community.
However manual memory management can lead to potential bugs if not handled carefully.\\
Assembly Language provides direct control over hardware and is highly efficient, and useful for writing low-level code, such as interrupt service routines but has a steeper learning curve and it is less portable across different microcontroller architectures.\\
C++ Language is an object-oriented feature that can enhance code organization and reusability and can provide abstraction without sacrificing performance.
However, the code size can be larger and more complex.\\

Real-Time Operating System facilitates multitasking, allowing concurrent execution of multiple tasks, can provide task scheduling, priority management, and inter-process communication and it is suitable for systems with real-time requirements.
But this can add overhead, especially in terms of memory footprint, and the learning curve is steeper \cite{RTOS1}.\\
The following \gls{RTOS} examples are open-source, well-documented, compact, and designed for resource-constrained systems, and they support various microcontroller architectures
\begin{itemize}
    \item FreeRTOS
    \item ChibiOS
    \item Zephyr
\end{itemize}
%-------------------------------------------------------------------------------%

\section{Communication}
\subsection{Wired}
\subsection{Wireless}
While researching wireless communication, multiple protocols can be studied. They can, mainly, be separated into two categories: short-range and long-range.
In the context of \glspl{uav}, the focus will be on short-range wireless communication protocols \cite{WCOM1}, \cite{WCOM6}, \cite{WCOM7}.

Short-range protocols offer advantages such as lower power consumption, reduced interference, and efficient data transfer within confined spaces.
Within this category, options like Bluetooth, Wi-Fi, Zigbee, Z-Wave, and LoRa for short distances emerge as noteworthy candidates.
Each of these protocols addresses specific requirements, making them suitable for various aspects of \gls{uav} operations, from intra-component communication to data transfer between the \gls{uav} and ground control \cite{WCOM6}, \cite{WCOM7}.

\begin{itemize}
    \item WiFi (802.11x): Can be used for high-speed data transfer over short ranges. It's suitable when you need to transmit large amounts of data between the \gls{uav} and a ground station.
        \begin{itemize}
            \item Advantages
                \begin{itemize}
                    \item High Data Rates
                    \item Widespread Standard
                    \item Bi-Directional Communication
                \end{itemize}
            \item Disadvantages 
                \begin{itemize}     
                    \item High Power Consumption
                    \item Interference in 2.4 GHz and 5 GHz bands
                \end{itemize}
        \end{itemize}
        
\item Bluetooth: Common short-range wireless technology with low power consumption. It's suitable for communication between components on a \gls{uav}.
    \item Advantages 
        \begin{itemize}
            \item Low Power Consumption
            \item Ubiquity
        \end{itemize}
    \item Disadvantages 
        \begin{itemize}
            \item Limited Range
            \item Data Transfer Rates
        \end{itemize}
        
\item Zigbee: Low-power, low-data-rate wireless communication technology that is suitable for short-range communication in embedded systems.
    \item Advantages 
        \begin{itemize}
            \item Low Power Consumption
            \item Mesh Networking
            \item Low Latency
        \end{itemize}
    \item Disadvantages
        \begin{itemize}
            \item Limited Data Rate
            \item Limited Range
        \end{itemize}
        
\item Z-Wave: Low-power wireless communication protocol often used in home automation. It's suitable for control and monitoring applications in \glspl{uav}.
    \item Advantages 
        \begin{itemize}
            \item Low Power Consumption
            \item Interference Avoidance
        \end{itemize}
    \item Disadvantages 
        \begin{itemize}
            \item Limited Data Rate
            \item Less Common in Non-Home Automation Devices
        \end{itemize}
        
\item LoRa (Long Range): While designed for long-range communication, LoRa can also be used in short-range applications. It provides low-power, long-range communication suitable for certain \gls{uav} scenarios.
    \item Advantages 
        \begin{itemize}
            \item Long Range
            \item Low Power Consumption
        \end{itemize}
    \item Disadvantages
        \begin{itemize}
            \item Low Data Rates
            \item Unidirectional Communication
        \end{itemize}
\end{itemize}