%%%%%%%%%%%%%%%%%%%%%%%%%%%%%%%%%%%%%%%
\section{\acs{CoAP} Security}
\label{sec:report}
%%%%%%%%%%%%%%%%%%%%%%%%%%%%%%%%%%%%%%%


\ac{CoAP} is a protocol designed for resource-constrained devices and networks which uses \ac{UDP} as its transport layer with RESTful protocol.
This protocol provides methods for resource discovery, retrieval, and manipulation. It also provides features for reliable message exchange, resource observation, and group communication. 
This allows clients to receive updates from servers when the resources change and can be very useful for monitoring data from sensors or other devices in real-time.

\acs{CoAP} is designed to be a very efficient protocol that minimizes the use of network resources, making it ideal for devices that have limited processing power, memory, and battery life.
In terms of security, \acs{CoAP} provides several security mechanisms (with different security levels) to ensure secure communication between devices. Some examples are:
\begin{itemize}
    \item \ac{DTLS}: Can be use to provide end-to-end security for data transmission between devices since it provides encryption, authentication, and integrity protection of messages.
    \item Resource access control: The protocol allows for access control of resources based on user identity and permissions.
    \item Secure group communication: Secure multicast and group communication.
    \item Security profiles: \acs{CoAP} defines several security profiles that can be used to specify security requirements for different application scenarios including \ac{PSK} and \ac{CBOR}.
\end{itemize}

There are documentations developed through a collaborative process and maintained by the Internet Engineering Task Force (IETF) called \ac{RFC} that describe standards, protocols, and best practices for the Internet and other computer networks.

\acs{RFC}s are essentially technical documents that describe how different Internet protocols and technologies work, as well as provide recommendations and guidelines for their implementation and usage. 
They can cover a wide range of topics, including network architecture, routing, security, email protocols, web protocols, and more.

\acs{RFC}s are assigned unique identification numbers, which are used to reference and cite them in other technical documents, academic papers, and other publications.

\acs{CoAP} also has \acs{RFC} documentation to describe the features, behavior, guidelines and best practices of the protocol
Here are some of the relevant \acs{RFC}s related to \acs{CoAP}:

\begin{itemize}
    \item \acs{RFC} 7228: 
    \item \acs{RFC} 7252: This RFC defines the \acs{CoAP} basic features including its message format, request/response methods, Uniform Resource Identifier mapping, and transport bindings
    \item \acs{RFC} 7641: This RFC provides guidelines for the use of CoAP in the Internet of Things (IoT) and machine-to-machine (M2M) applications.
    \item \acs{RFC} 7662:
    \item \acs{RFC} 7959: Defines the Block-wise Transfers in \acs{CoAP}, which is a mechanism to allow large payloads to be split into multiple smaller messages and transferred over networks with limited resources.
    \item \acs{RFC} 8075: This RFC defines the usage of CoAP over TCP, which is an alternative transport protocol for CoAP messages.
    \item \acs{RFC} 8323: This RFC defines the CoAP Management Interface (CoMI), which provides a standardized interface for managing CoAP resources.
    \item \acs{RFC} 8613: This RFC defines the CoAPs protocol, which is a secure version of CoAP that uses Datagram Transport Layer Security (DTLS) to provide end-to-end security.
    \item \acs{RFC} 8890: This RFC specifies the CoAP Content-Format Indicators, which provide a standardized way of indicating the format of CoAP payloads.
\end{itemize}
These \acs{RFC}s are important references for developers who want to implement CoAP-based applications or protocols. They provide a comprehensive understanding of the protocol and its use cases.

Like \acs{MQTT}, there is also several \acs{CVE}s related to \acs{CoAP} implementations. Here are some examples:
\begin{itemize}
    \item CVE-2018-18913: A vulnerability in Contiki OS that allows a remote attacker to execute arbitrary code or cause a \ac{DoS}.
    \item CVE-2019-12415: A vulnerability in Eclipse Californium that could allow a remote attacker to cause a \ac{DoS} condition by sending a specially crafted \acs{CoAP} packet with a large block option value.
    \item CVE-2020-15523: A vulnerability in TinyOS which allows a remote attacker to cause a buffer overflow and execute arbitrary code.
\end{itemize}


Even through \acs{CoAP} and \acs{MQTT} are both protocols used for communication, they have different security mechanisms due to their design and requirements.

Overall, both \acs{CoAP} and \acs{MQTT} provide security features that can be used to protect device communications and data. 
However, the choice between the two protocols may depend on the specific security requirements of the application and the level of support for security mechanisms.

\clearpage