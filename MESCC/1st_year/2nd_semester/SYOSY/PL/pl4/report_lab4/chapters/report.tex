%%%%%%%%%%%%%%%%%%%%%%%%%%%%%%%%%%%%%%%
\section{Communication Protocol Comparison}
\label{sec:report}
%%%%%%%%%%%%%%%%%%%%%%%%%%%%%%%%%%%%%%%

In the last classes we learned and explored \ac{MQTT}, \ac{CoAP} and \ac{DDS}.
These three different protocols have its own strengths and weaknesses and are mainly used in the field of the \ac{IoT} and \ac{M2M} communications.

Here is a brief explanation:
\begin{itemize}
    \item \acs{MQTT}: Publish-subscribe messaging protocol that is designed for low-bandwidth networks. 
    Uses Transport Layer Security (TLS) for encryption.
    It uses a broker to route messages between publishers and subscribers. 
    Standardized by the IETF.
    \item \acs{CoAP}: Request-response protocol that supports caching, multicast and observe features. 
    Uses Datagram Transport Layer Security (DTLS) for encryption.
    It has a client-server architecture and it is designed for low-power wireless networks. 
    Standardized by the IETF.
    \item \acs{DDS}: Data-centric publish-subscribe messaging protocol that supports the exchange of large amounts of data between different devices. 
    Supports multiple security mechanisms, including \acs{TLS}, \acs{DTLS} and \acs{SRTP}.
    It is designed for real-time systems.
    Standardized by the Object Management Group.
\end{itemize}

All three protocols are lightweight and efficient in terms of bandwidth and power consumption, they use asynchronous messaging with different levels of \acs{QoS}, in all three of them it is easy to add or remove devices maintaining the architecture and are platform-independent protocols which can be used in various operating systems and hardware.

Comparing them in terms of architecture, \acs{MQTT} and \acs{DDS} use publish-subscribe architecture, where the publisher sends data to a broker, which routes the data to all interested subscribers.
While \acs{CoAP} uses a client-server architecture in which the client sends a request to a server, which sends a response back.

When it comes to network requirements and message size, \acs{MQTT} and \acs{CoAP} are designed to be implemented in networks with limited bandwidth and have a limit on the message size, typically around 256 bytes.
\acs{DDS}, on the other hand, can handle large amounts of data making it suitable for high-bandwidth networks.

Even thought all three protocols support security features such as authentication and encryption, \acs{DDS} provides more advanced security features such as access control, data encryption and secure discovery.

\acs{MQTT} and \acs{CoAP} support different levels of \ac{QoS} for message delivery (three levels in \acs{MQTT} and four in \acs{CoAP}).
In contrast, \acs{DDS} supports configurable \acs{QoS} levels based on the importance of the data.

It is important to notice that \acs{DDS} is a middleware protocol, while \acs{MQTT} and \acs{CoAP} are application layer protocols.
Meaning that \acs{DDS} can provide more advanced features such as content-based filtering, data caching, and distributed coordination.

Each of the studied protocol can be used in a variety of different applications depending on the specific requirements and use case.

\acs{MQTT} protocol gives the functionality to remote control and monitor the system:
\begin{itemize}
    \item Smart Home automation: Connecting various smart home devices.
    \item Industrial Automation: Connecting industrial devices such as sensors, PLCs and HMIs.
    \item Remote sensing: Connecting central servers to remote sensors like weather stations and environmental monitoring devices.
\end{itemize}

Likewise, \acs{CoAP} can also give the functionality to remote control and monitor systems that, in the case of this protocol, are low-power:
\begin{itemize}
    \item Smart city applications: Connecting various smart city devices such as streetlights, parking sensors and waste management systems.
    \item Healthcare monitoring: Connecting central servers to wearable devices such as fitness trackers and medical monitors.
    \item Building automation: Connecting various building automation systems such as lighting, HVAC and security systems.
\end{itemize}

\acs{DDS} should be used is applications where it is important to have real-time decision making, coordination, situational awareness and monitoring:
\begin{itemize}
    \item Autonomous Vehicles: Connecting various systems in an autonomous vehicle such as sensors, actuators and control systems.
    \item Aerospace and Defense: Connecting various systems in an aerospace or defense application like radar systems, communication systems and control systems.
    \item Medical equipment: Connecting various medical equipment, for example patient monitors, ventilators and infusion pumps.
\end{itemize}
\clearpage