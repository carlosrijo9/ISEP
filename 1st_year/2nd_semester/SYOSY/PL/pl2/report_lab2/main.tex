% ////////////////////////////////////////////////////////////////////////
% ///
% /// @author Carlos Rijo <1101626@isep.ipp.pt>
% ///
% /// @date 05-03-2023
% ///
% ////////////////////////////////////////////////////////////////////////

\documentclass[a4paper,11pt,twoside]{article}

\usepackage[sfdefault,light]{FiraSans}              % option 'sfdefault' activates Fira Sans as the default text font
\usepackage[T1]{fontenc}
\renewcommand*\oldstylenums[1]{{\firaoldstyle #1}}

\usepackage[numbers]{natbib}	                    % use natbib for the references

\usepackage{makeidx}	                            % Creating indexes
\usepackage{graphicx,wrapfig}	                    % Importing graphics
\graphicspath{{images/}}                            %Setting the graphics path

\usepackage[a4paper,margin=1in]{geometry}	        % Page layout
\usepackage{listings}	                            % Source code snippets
\usepackage{enumitem}
\usepackage{fancyvrb}
\usepackage{fancyhdr}

\setlength{\headheight}{30pt}
\def\code#1{\texttt{#1}}
\usepackage{lastpage}
\usepackage{textcomp}
\usepackage{amssymb}
\usepackage{amsmath}
\usepackage{floatrow}
\usepackage[printonlyused]{acronym}
\usepackage{caption}
\usepackage{subcaption}
\usepackage{sfmath}
\usepackage{siunitx}
\usepackage[hidelinks]{hyperref}	               % Hypertext marks
\usepackage{bookmark}
\usepackage[toc,page]{appendix}
\usepackage[nottoc]{tocbibind}
\usepackage{parskip}
\usepackage[dvipsnames]{xcolor}
\usepackage{array}                                 % For tabular alignment
% \usepackage[backend=biber, style=ieee, sorting=none]{biblatex}
% \addbibresource{bibliography.bib}
\usepackage{longtable}
\usepackage{makecell}
\usepackage{multirow}
\usepackage{tikz}
\usepackage[siunitx]{circuitikz}
\usepackage{verbatim}
\usepackage{booktabs}% http://ctan.org/pkg/booktabs

\newcommand{\tabitem}{~~\llap{\textbullet}~~}

\newcommand{\todo}[1]{\textcolor{red}{#1}}

\definecolor{ceiiablue}{RGB}{0, 185, 225}
\definecolor{cultured}{RGB}{245, 245, 245}
\definecolor{olivegreen}{RGB}{166, 168, 103}

% Table float box with bottom caption, box width adjusted to content
\newfloatcommand{capbtabbox}{table}[][\FBwidth]

% Define a new column type with horizontal centering
\newcolumntype{P}[1]{>{\centering\arraybackslash}p{#1}}

\DeclareCaptionFont{captiontext}{\color{black}}
\DeclareCaptionFont{captionlabel}{\bfseries\color{ceiiablue}}
\DeclareCaptionFormat{listing}{\colorbox{white}{\parbox{\dimexpr\linewidth-2\fboxsep\relax}{#1#2#3}}}

\captionsetup[figure]{labelfont=captionlabel,textfont=captiontext}
\captionsetup[table]{labelfont=captionlabel,textfont=captiontext}
\captionsetup[lstlisting]{labelfont=captionlabel,textfont=captiontext}

\lstset{ 
  backgroundcolor=\color{cultured},  % choose the background color
  basicstyle=\footnotesize\ttfamily, % the size of the fonts that are used for the code
  breakatwhitespace=false,           % sets if automatic breaks should only happen at whitespace
  breaklines=true,                   % sets automatic line breaking
  commentstyle=\color{olivegreen},   % comment style
  extendedchars=true,                % lets you use non-ASCII characters; for 8-bits encodings only, does not work with UTF-8
  keepspaces=true,                   % keeps spaces in text, useful for keeping indentation of code (possibly needs columns=flexible)
  keywordstyle=\color{ceiiablue},    % keyword style
  showspaces=false,                  % show spaces everywhere adding particular underscores; it overrides 'showstringspaces'
  showstringspaces=false,            % underline spaces within strings only
  showtabs=false,                    % show tabs within strings adding particular underscores
  tabsize=2 	                     % sets default tabsize to 2 spaces
}

\makeindex
%%%%%%%%%%%%%%%%%%%%%%%%%%%%%%%%%%%%%%%
% User defined functions and variables
%%%%%%%%%%%%%%%%%%%%%%%%%%%%%%%%%%%%%%%
\newcommand{\HRule}{\rule{\linewidth}{0.3mm}}                       % Defines a new command for the horizontal lines, change thickness here
\newcommand{\Version}{1.0}
\newcommand{\Logo}{logo.jpg}
\newcommand{\Product}{cover.jpg}
\newcommand{\Title}{Assignment 2}
\newcommand{\Subtitle}{Report}
\newcommand{\DocumentDate}{Mar 15, 2023}


%%%%%%%%%%%%%%%%%%%%%%%%%%%%
% Header / Footer
%%%%%%%%%%%%%%%%%%%%%%%%%%%%
\pagestyle{fancy}
\fancyhead{}                                                        % Clear all header fields
\fancyhead[L]{\includegraphics[height=25pt]{\Logo}}
\fancyhead[C]{\Title - \Subtitle}
%\fancyhead[R]{v\Version}
\fancyfoot{}                                                        % Clear all footer fields
\fancyfoot[C]{\thepage/\pageref{LastPage}}


%%%%%%%%%%%%%%%%%%%%%%%%%%%%

\begin{document}

\begin{titlepage}
    %\hspace*{-0.7cm}\includegraphics[width=0.25\linewidth]{\Logo}\\ %  logo
    %[1cm]
    \textcolor{ceiiablue}{\textbf{\Huge \Title}}\\                  % Title
    \HRule \\[0.2cm]
    \textsc{\Large \Subtitle}                                       % Subtitle
    \vfill
    \begin{center}
        \includegraphics[width=0.8\linewidth]{\Product}
    \end{center}

    {\raggedleft\vfill\itshape{
        Carlos Rijo - 1101626
        \newline
        \DocumentDate}
    }
\end{titlepage}

% \section*{Version History}
% \begin{center}
%     \begin{tabular}{c|c|P{3cm}|p{7cm}}
%         \textbf{Version} & \textbf{Date} & \textbf{Author} & \textbf{Description}                     \\\hline
%         1.0              & 2023-02-20    & CAR             & Created the first issue of the document. \\
%     \end{tabular}
% \end{center}
\clearpage

\tableofcontents
\newpage
% \listoffigures
% \newpage
% \listoftables
% \newpage
%%%%%%%%%%%%%%%%%%%%%%%%%%%%%%%%%%%%%%%
\section*{List of Acronyms and Definitions}
\addcontentsline{toc}{section}{List of Acronyms and Definitions}
%%%%%%%%%%%%%%%%%%%%%%%%%%%%%%%%%%%%%%%
\begin{acronym}
    \acro{ISEP}{Instituto Superior de Engenharia do Porto}
    \acro{DEI}{Departamento de Engenharia de Informatica}
    \acro{MESCC}{Mestrado de Engenharia Sistemas Computacionais Criticos}
    \acro{SYOSY}{Systems of Systems}
    \acro{MQTT}{Message Queuing Telemetry Transport}
    \acro{CoAP}{Constrained Application Protocol}
    \acro{QoS}{Quality of Service}
    \acro{TLS}{Transport Layer Security}
    \acro{IoT}{Tnternet of Things}
    \acro{M2M}{Machine-to-Machine}
    \acro{CVE}{Common Vulnerabilities and Exposures}
    \acro{DoS}{Denial of Service}
    \acro{MitM}{Man-in-the-Middle}
    \acro{UDP}{User Datagram Protocol}
    \acro{DTLS}{Datagram Transport Layer Security}
    \acro{SRTP}{Secure Real-time Transport Protocol}
    \acro{CBOR}{Certificate-Based Authentication}
    \acro{PSK}{Pre-Shared Key}
    \acro{RFC}{Request for Comments}
    \acro{DDS}{Data Distribution Service}
\end{acronym}


%%%%%%%%%%%%%%%%%%%%%%%%%%%%%%%%%%%%%%%%%%%%%%%%%%%%%%%%%%%%%%%%%%%%%
% Having sections/chapters in their own files keeps things organized.
%%%%%%%%%%%%%%%%%%%%%%%%%%%%%%%%%%%%%%%%%%%%%%%%%%%%%%%%%%%%%%%%%%%%%

% \input does not use a \clearpage before and after content, \include will.
% IMPORTANT! do not write the .tex file extension when using \include!

%\include{./chapters/overview}
%%%%%%%%%%%%%%%%%%%%%%%%%%%%%%%%%%%%%%%
\section{Communication Protocol Comparison}
\label{sec:report}
%%%%%%%%%%%%%%%%%%%%%%%%%%%%%%%%%%%%%%%

In the last classes we learned and explored \ac{MQTT}, \ac{CoAP} and \ac{DDS}.
These three different protocols have its own strengths and weaknesses and are mainly used in the field of the \ac{IoT} and \ac{M2M} communications.

Here is a brief explanation:
\begin{itemize}
    \item \acs{MQTT}: Publish-subscribe messaging protocol that is designed for low-bandwidth networks. 
    Uses Transport Layer Security (TLS) for encryption.
    It uses a broker to route messages between publishers and subscribers. 
    Standardized by the IETF.
    \item \acs{CoAP}: Request-response protocol that supports caching, multicast and observe features. 
    Uses Datagram Transport Layer Security (DTLS) for encryption.
    It has a client-server architecture and it is designed for low-power wireless networks. 
    Standardized by the IETF.
    \item \acs{DDS}: Data-centric publish-subscribe messaging protocol that supports the exchange of large amounts of data between different devices. 
    Supports multiple security mechanisms, including \acs{TLS}, \acs{DTLS} and \acs{SRTP}.
    It is designed for real-time systems.
    Standardized by the Object Management Group.
\end{itemize}

All three protocols are lightweight and efficient in terms of bandwidth and power consumption, they use asynchronous messaging with different levels of \acs{QoS}, in all three of them it is easy to add or remove devices maintaining the architecture and are platform-independent protocols which can be used in various operating systems and hardware.

Comparing them in terms of architecture, \acs{MQTT} and \acs{DDS} use publish-subscribe architecture, where the publisher sends data to a broker, which routes the data to all interested subscribers.
While \acs{CoAP} uses a client-server architecture in which the client sends a request to a server, which sends a response back.

When it comes to network requirements and message size, \acs{MQTT} and \acs{CoAP} are designed to be implemented in networks with limited bandwidth and have a limit on the message size, typically around 256 bytes.
\acs{DDS}, on the other hand, can handle large amounts of data making it suitable for high-bandwidth networks.

Even thought all three protocols support security features such as authentication and encryption, \acs{DDS} provides more advanced security features such as access control, data encryption and secure discovery.

\acs{MQTT} and \acs{CoAP} support different levels of \ac{QoS} for message delivery (three levels in \acs{MQTT} and four in \acs{CoAP}).
In contrast, \acs{DDS} supports configurable \acs{QoS} levels based on the importance of the data.

It is important to notice that \acs{DDS} is a middleware protocol, while \acs{MQTT} and \acs{CoAP} are application layer protocols.
Meaning that \acs{DDS} can provide more advanced features such as content-based filtering, data caching, and distributed coordination.

Each of the studied protocol can be used in a variety of different applications depending on the specific requirements and use case.

\acs{MQTT} protocol gives the functionality to remote control and monitor the system:
\begin{itemize}
    \item Smart Home automation: Connecting various smart home devices.
    \item Industrial Automation: Connecting industrial devices such as sensors, PLCs and HMIs.
    \item Remote sensing: Connecting central servers to remote sensors like weather stations and environmental monitoring devices.
\end{itemize}

Likewise, \acs{CoAP} can also give the functionality to remote control and monitor systems that, in the case of this protocol, are low-power:
\begin{itemize}
    \item Smart city applications: Connecting various smart city devices such as streetlights, parking sensors and waste management systems.
    \item Healthcare monitoring: Connecting central servers to wearable devices such as fitness trackers and medical monitors.
    \item Building automation: Connecting various building automation systems such as lighting, HVAC and security systems.
\end{itemize}

\acs{DDS} should be used is applications where it is important to have real-time decision making, coordination, situational awareness and monitoring:
\begin{itemize}
    \item Autonomous Vehicles: Connecting various systems in an autonomous vehicle such as sensors, actuators and control systems.
    \item Aerospace and Defense: Connecting various systems in an aerospace or defense application like radar systems, communication systems and control systems.
    \item Medical equipment: Connecting various medical equipment, for example patient monitors, ventilators and infusion pumps.
\end{itemize}
\clearpage
\clearpage
% \renewcommand*{\bibname}{References}
% \bibliographystyle{./other/IEEEtranN}
% \bibliography{./other/bibliography}

\end{document}
